\documentclass[a4paper]{article}

\usepackage{hyperref}

\title{\textbf{Robotics Summerschool Juli 2012} \\ Opdrachten Dag 3}
\author{Dutch Nao Team - \url{http://dutchnaoteam.nl}}
\date{}

\begin{document}
\maketitle

\section{Introductie}
Welkom bij de Summerschool Robotics.\\
\\
De nodige code voor deze opdracht kun je downloaden van:\\ \url{http://ducthnaoteam.nl/summerschool/dag3\_start.zip}\\
\\
Unzip de code in de map \textit{SummerschoolRobotics2012} op je desktop. Je kunt nu beginnen met de derde opdracht van deze summerschool. \\
\\
Veel plezier en succes!\\
\\
Het Dutch Nao Team.\\
\\
\textbf{Wireless informatie:}\\
Router:  Dutch Nao Team Robolab\\
Wachtwoord: DnT10RuleS\\
Deze router is voor de mensen, zittend het dichts bij de lift.\\
\\
Router:  Dutch Nao Team NAO\\
Wachtwoord: sam+moos\\
Deze router is voor de mensen, zittend het dichts bij de glazen wand


\tableofcontents

\newpage


\section{Opdrachten}
Voor deze opdracht is het de bedoeling een pathplanner met behulp van A* ontwikkeld wordt die de mogelijkheid geeft de Nao een weg door het doolhof te laten vinden. Daarbij zijn er de volgende moeilijkheidsgraden beschikbaar:

\subsection{Moeilijk}
Implementeer geheel zelf een pathplanner met behulp van A*. Redeneer hierbij voor je begint te programmeren zorvuldig over de volgende aspecten:
\begin{itemize}
\item Hoe representeer je op een slimme manier verschillende states van het doolhof?
\item Hoe bepaal je wat een geldige zet (move) is in het doolhof?
\item Wat gebruik je als kostenfunctie bij het kiezen van elke zet?
\item Wat is de uiteindelijke goalstate die je probeert te bereiken?
\item Hoe representeer je de output, ofwel het pad dat doorlopen moet worden?
\end{itemize}

Probeer modulair te werken. Maak aparte functies die controleren of het doel is bereikt, wat de kosten
zijn van een bepaalde state transitie, enzovoorts.

Als alles geimplementeerd is, maak een main functie om je code uit te testen. Probeer verschillende doolhoven uit als je representatie zich hiervoor leent.\\
Optioneel: maak een mooie visuele presentatie van hoe het doolhof doorlopen moet worden.


\subsection{Middelmatig/moeilijk}
Werk in het bestand \textit{aStar\_mediummoeilijk.py}.\\
De representatie van het doolhof is hier al gegeven. Zie hiervoor het bestand \textit{maze.txt} voor een mogelijke voorbeeldrepresentatie. Het bestand \textit{mazeParser.py} maakt hier een leesbare output van. De variabele validMove die in de main van de file \textit{aStar\_mediummoeilijk.py} wordt aangemaakt bevat een dictionary met in alle posities van het doolhof de geldige posities die hier aan grenzen (de mogelijke moves vanaf elke state zijn dus eigenlijk al bekend).

Hierna moet zelf aan de slag gegaan worden met deze validMove dictionary om A* in te implementeren. 

\subsection{Middelmatig}
Werk in het bestand \textit{aStar\_medium.py}.\\
De representatie van het doolhof is hier al gegeven. Zie hiervoor het bestand maze.txt voor een mogelijke voorbeeldrepresentatie. 
Het bestand mazeParser.py maakt hier een leesbare output van. De variabele validMove die in de main van de file aStar\_medium.py wordt aangemaakt bevat een dictionary met in alle posities van het doolhof de geldige posities die hier aan grenzen (de
mogelijke moves vanaf elke state zijn dus eigenlijk al bekend).

\textit{findShortestPath} is hier de functie waarin A* geimplementeerd moet worden. Een beginnetje is al gemaakt, maar de loop moet nog zelf geimplementeerd worden. De opzet van functies finished((estimatedCost, costSoFar, path), end) en getCost(node1,
node2) zijn gegeven maar moeten worden ingevuld en verwerkt in je algoritme.

\subsection{Makkelijk}
Werk in het bestand \textit{aStar\_makkelijk.py}.\\

De meeste functies van de A* implementatie zijn al gegeven, waaronder de main loop. Er moeten wel nog wat functies ingevuld worden, waaronder:
\textit{
\begin{itemize}
\item getUnseenNeighbors(seen, neighbors)
\item addToQueue(queue, newPaths)
\item finished((estimatedCost, costSoFar, path), end)
\item getCost(node1, node2)
\end{itemize}
}

Lees de comments bij elke functie header om erachter te komen wat er wordt verwacht van de functie en 
pas dit aan. Voordat je begint is het verstandig eerst aandachtig naar de volgende functies te kijken:
\begin{itemize}
\item findShortestPath(start, end, validMoves)
\item getNewPaths(estimatedCost, costSoFar, path, lastNode, end, neighbors)
\end{itemize}

Kijk of je alle input en output waardes goed begrijpt. Begin daarna pas met programmeren.


\end{document}
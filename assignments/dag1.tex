\documentclass[dutch, a4paper]{article}


\begin{document}
\tableofcontents

\newpage

\section{Introductie}
Welkom bij de summerschool

Download de code van .....
Unzip het 
\\\\\\\\\\\\\\\\\\\\\\\\\\\\\\\\



\section{Opdrachten}

\subsection{Python DNT framework Hello World}
Komende dagen zul je gaan werken binnen een framework, bestaande uit verschillende modules. Om hier goed gebruik van te kunnen maken is het belangrijk dat je de bijgevoegde readme leest (zie appendix) en de volgende drie dingen en hun functies kent:
\begin{itemize}
\item config.py - start de code
\item module - bevat minimaal de functie \textit{setDependencies(self, modules)}
\item main-module - bevat minimaal de functies \textit{setDependencies(self, modules)} en \textit{start(self)}
\end{itemize}

\subsubsection{Een Main-Module aanmaken}
Open een nieuw document in Notepad++, noem het \textit{main.py} en sla het op in de map \textit{modules} van het geleverde framework. 
Zoals beschreven in de readme, begint een module met een klasse-naam gelijk aan zijn bestandsnaam. Een main-file bevat daarnaast ook minimaal 2 functies: \textit{setDependencies} en \textit{start}.
Omdat de functies op dit moment nog geen inhoud hebben, moet je het commando \textit{pass} onder de functie-declaratie geven. Het bestand zou er nu als volgt uit moeten zien:

\noindent \line(1,0){100}
\begin{verbatim}
class main:
    def setDependencies(self, modules):
        pass

    def start(self):
        pass
\end{verbatim}
\noindent \line(1,0){100}
\\\\
Vergeet niet het bestand op te slaan!

\subsubsection{Een module voor globale variabelen}
Binnen het framework zullen meerdere variabelen straks gedeeld worden door meerdere modules. Daarom is er een module nodig die deze variabelen en functies bevat.

Open een nieuw document in Notepad++ en noem het \textit{globals.py}, sla het op in de map \textit{modules} en geef het de juiste klasse-naam en functie(s). 

Een robot wordt aangeroepen aan de hand van zijn ipadress. Start je Nao op en achterhaal zijn ipadress op door op zijn chest-button te drukken. Dit ipadress moet als string worden genoteerd in de globals-module:

\noindent \line(1,0){100}
\begin{verbatim}
class globals:
    ipadress = "<ipadress-van-je-nao>"

    def setDependencies(self, modules):
        pass
\end{verbatim}
\noindent \line(1,0){100}
\\\\
Binnen de Naoqi (de software waar de Nao op draait) worden naoqi-modules aangeroepen via een proxy: ALProxy (\textit{from naoqi import ALProxy}). Om slechts 1 keer verbinding te maken met alle proxies defineren we een functie, \textit{setProxies(self)}, die dit doet. Deze functie zal straks door de main-module worden aangeroepen.

De Nao beschikt over een TextToSpeech-module, deze is aan te roepen doormiddel van: \textit{ALProxy``ALTextToSpeech", self.ipadress, 9559)}
en zet tekst om in spraak. Andere modules in ons framework moeten deze naoqi-module kunnen aanroepen, dus slaan we de zojuist gemaakte referentie op in de variabele \textit{self.speechProxy}.

Je code zou er nu als volgt uit moeten zien:

\noindent \line(1,0){100}
\begin{verbatim}
from naoqi import ALProxy
class globals:
    ipadress = "192.168.0.1"

    def setDependencies(self, modules):
        pass

    def setProxies(self):
        self.speechProxy = ALProxy("ALTextToSpeech", self.ipadress, 9559)
\end{verbatim}
\noindent \line(1,0){100}

\subsubsection{modules registreren in \textit{config.py}}
De zojuist gemaakte modules moeten nu nog worden geregistreerd in het framework. Hiervoor is er een bestand config.py, open dit in Notepad++.

Als eerste moet de main-module worden geregistreerd. Zoals in de readme beschreven, wordt dit gedaan door:
 \textit{moduledict[``main"] = ``$<$class-name-of-module$>$" }. 
Omdat wij onze main-module \textit{main} hebben genoemd, ziet de registratie er uit als: \textit{moduledict[``main"] = ``main"}.
Registratie wordt gedaan vanaf regel 17 en voor het stuk dat het framework importeert en uitvoert.

Ook de globals-module moet worden geregistreerd. Om overzicht te houden is het raadzaam om de referentie naam \textit{``globals"} te noemen. Aan de hand van deze naam kunnen straks andere modules globals aanroepen en opvragen. De module naam is ook ``globals", daarom ziet deze registratie er als volgt uit: \textit{moduledict[``globals"] = [1, ``globals"]}

\textit{config.py} zou er nu ongeveer zo uit moeten zien:

\noindent \line(1,0){100}
\begin{verbatim}
# @file config.py
# @func module definitions for framework. 
#       Is used to register, define & load modules 
#       Starts the main-framework
# @auth Hessel van der Molen
#       hmolen.science@gmail.com
# @date 4 may 2012

#dictionary for module
moduledict = {}
#show frameworks print's (1=show, 0=do not show)
VERBOSE = 1

####
## Register modules:
#######

moduledict["main"] = "main"
moduledict["globals"] = [1, "globals"]

###########################
# start & run framework
###########################
from framework import mframework
mframework.startUpFramework(moduledict, VERBOSE)
\end{verbatim}
\noindent \line(1,0){100}

\subsubsection{Het runnen van de code}
Het framework (en onze code) is nu klaar om te runnen. Dit wordt gedaan door het programma \textit{Command Prompt} (All Programs $>$ Accessories) te openen. 
Ga nu behulp van de commando's ``cd"  en ``dir" naar de map waar \textit{config.py} staat.
Door \textit{python config.py} te typen wordt de code gestart.

Start de code en bekijk wat er gebeurd.

Als het goed is, sluit het systeem (na enkele prints te hebben gedaan) netjes af. Dit klopt, aangezien onze main nog geen aanroep maakt of code uitvoert.

Om het Hello World-voorbeeld af te maken moeten we weer terug gaan naar onze main-module.

De main-module is afhankelijk (dependent) van de globals. Daarom moet de \textit{globals}-module via \textit{setDependencies(self, modules)} worden aangeroepen:

\noindent \line(1,0){100}
\begin{verbatim}
def setDependencies(self, modules):
    self.globals = modules.getModule("globals")
\end{verbatim}
\noindent \line(1,0){100}
\\\\
Via de variable \textit{self.globals} hebben we in de main-module toegang tot alle globale data.

Om iets met de robot te kunnen doen moeten de proxies worden gezet met \textit{self.globals.setProxies()} nadat onze code is opgestart (na \textit{start(self)} is aangeroepen door het framework).
Wanneer de proxies zijn aangeroepen kunnen we gebruik maken van de variabele \textit{speechProxy} uit \textit{globals}.
Met behulp van het commando \textit{say(``Text...")} kan de Nao text uitspreken. In ons systeem ziet dit er als volgt uit:

\noindent \line(1,0){100}
\begin{verbatim}
def start():
    self.globals.setProxies()
    self.globals.speechProxy.say("Hello World")
\end{verbatim}
\noindent \line(1,0){100}
\\\\
Sla alle code op en run opnieuw \textit{config.py}.

\subsection{Lopen, Vallen en Opstaan}
\indent \\\\\\\\\\\\\\\\\\\\\\\\\\\\\\\\







\subsection{Loop-parameters}
\indent \\\\\\\\\\\\\\\\\\\\\\\\\\\\\\\\











\subsection{Tracking}
\indent \\\\\\\\\\\\\\\\\\\\\\\\\\\\\\\\


\subsection{\indent \indent}
\indent \\\\\\\\\\\\\\\\\\\\\\\\\\\\\\\\


\section{appendix}
\end{document}